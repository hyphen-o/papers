%
% 卒論レジュメフォーマット Ver.2.0 pLaTeX版
%
\documentclass[twocolumn]{jarticle} % 2段組のスタイルを用いている

\usepackage{wuse_resume}
\usepackage{url}	% \url{}コマンド用.URLを表示する際に便利
%\usepackage[dvipdfmx]{graphicx}  % ←graphicx.styを用いてEPSを取り込む場合有効にする
			% 他のパッケージ・スタイルを使う場合には適宜追加

%%%%%%%%%%%%%%%%%%%%%%%%%%%%%%%%%%%%%%%%%%%%%%%%%%%%%%%%%%%%%%%%%%%%%%%%

%%
%% タイトル,学生番号,氏名などを設定する
%%

\タイトル{Scratchにおける学習者の作品制作過程に基づく\\
コンピュテーショナル・シンキング習熟度の到達予測}
\研究室{ソーシャルソフトウェア工学}
\学生番号{60256053}
\氏名{岡本  圭悟}

\概要{%
本研究では,Scratchにおけるユーザのコンピュテーショナル・シンキング(CT)の作品制作過程に合わせた作品推薦に向けて,ユーザが予測時点でCT習熟度が向上する作品を制作するか否かを予測する手法を提案する.
Scratchでは,視覚的に表現されたブロックを組み合わせてプログラムを実装し,作品制作を通してCTを身につける.特に,ユーザは学習支援ツールDr.Scratchを使用することで,自身のCTスキルを定量的に把握できる.
従来研究では,ユーザが過去に獲得したCTスキルから,特定のCT習熟度への到達可否を予測するモデルを構築,評価した.しかし,従来モデルではユーザの作品制作過程は十分考慮できておらず,誤って予測することがある.
本研究では,まずユーザが獲得してきたCT7概念の特徴量を分析し,ユーザの作品制作過程を説明変数とした学習モデルを構築し,従来モデルとの比較評価を行った.
}

\キーワード{Scratch}
\キーワード{プログラミング教育}
\キーワード{機械学習}

%%%%%%%%%%%%%%%%%%%%%%%%%%%%%%%%%%%%%%%%%%%%%%%%%%%%%%%%%%%%%%%%%%%%%%%%

%% 以下の3行は変更しない

\begin{document}
\maketitle
\thispagestyle{empty} % タイトルを出力したページにもページ番号を付けない

%%%%%%%%%%%%%%%%%%%%%%%%%%%%%%%%%%%%%%%%%%%%%%%%%%%%%%%%%%%%%%%%%%%%%%%%

%%
%% 本文 - ここから
%%

\section{はじめに}

ビジュアルプログラミング言語Scratch\cite{Resnick_2009}では,プログラミングにおける命令処理を視覚的なブロックとして表現し,それらを組み合わせることでプログラムが実装できる.また,ScratchはWeb上にオンライン学習サービス\footnote{Scratch: \url{https://scratch.mit.edu/}}を展開しており,ユーザは他ユーザが制作した作品を参照することが可能である.

ユーザはScratch上で作品制作を通してプログラミングに必要な能力であるコンピュテーショナル・シンキング(CT)を身につける.Scratch上でCTを定量的に把握できるツールとして,MorenoらはDr.Scratch~\cite{Moreno_2015}を開発している.Dr.Scratchは,Scratchプログラムから,作品の機能実装に必要な7つのCT概念をそれぞれ0点から3点までで算出し,合計点数0点から21点で作品を評価する.特に,CTスキルの区分としてCT習熟度が存在し,0点から7点をBasic,8点から14点をDeveloping,15点から21点をMasterとしている.

従来研究として安東ら~\cite{Ando_2021}はユーザのCTに合わせた作品推薦に向けて,ユーザが過去に制作した作品のCTスキルに基づいてユーザが次にCT習熟度が向上するかを予測する研究を行なった.しかし,従来モデルではユーザの作品製制作過程は十分考慮できておらず誤って予測することがある.このことから作品制作過程を考慮したモデルを構築することで,従来モデルで予測できなかった作品制作過程が類似するユーザの習熟度到達予測が可能になると考える.

% RQ1ではScratch作品の制作過程でCT習熟度が向上したユーザに着目して,各ユーザのCT7概念の獲得過程(CTパス)の特徴量を分析し,ユーザのCTパスの特徴,ユーザのCT概念の獲得過程に意味があるかを確認する.RQ2では学習者のCT7概念の獲得過程を考慮した学習支援のため,,ユーザのCTスコア獲得過程を説明変数として学習モデルを作成し,ユーザが特定の習熟度に到達するか否かの予測を行い,従来モデルとの比較評価を行う.

% 以降,本論文では,2章で本論文の対象であるビジュアルプログラミング環境Scratchとそれに関連する従来研究
% ,本研究の位置づけを述べ,3章,4章では,設定したRQにおけるそれぞれの提案手法と結果,考察を述べる.続く5章では,妥当性の脅威を述べ,最後に6章で本論文をまとめる.

\section{ユーザの作品制作過程に基づくCT習熟度到達タイミングの予測}

従来研究では,ユーザがCT習熟度を向上させる過程でどのようなCTスキルを獲得しているかは詳細に分析されていない.したがって,本分析ではまず作品制作過程とCT概念の特徴をそれぞれ分析する.そしてCT獲得過程を考慮した説明変数を提案し,モデルを構築し,従来モデルとの精度比較を行う.

\subsection{データセット}

Scratchサービス上に20件以上の作品を公開しているユーザ6,323人が1番目から20番目までに制作した作品126,460件を本分析の対象とする.

\subsection{事前分析}\label{sec:before-analyze}

本分析では,BasicからDeveloping以上に向上したユーザ(BtoDユーザ)とDevelopingからMasterに向上したユーザ(DtoMユーザ)を対象に,作品間のエッジの重複数と頻出するCTパスを分析した.

結果として,BtoDユーザは同一CT概念の作品を繰り返し制作する傾向があり,DtoMユーザは多様なCT概念の作品を制作することが明らかとなった.

\subsection{提案手法}

従来研究より,目的変数の異なる2種類の予測モデルを構築する.

\begin{description}
\item [BtoDモデル:]Developing以上のオリジナル作品を制作するユーザを予測
\item [DtoMモデル:]Masterのオリジナル作品を制作するユーザを予測
\end{description}

目的変数は,{$N~(3 \leq N \leq 20)$}番目にDeveloping以上またはMasterに到達するオリジナル作品を制作したユーザを正例クラス,それ以外のユーザを負例クラスとする.本実験では従来研究で用いた計56次元の説明変数に加え,ユーザの作品制作過程を捉えるための新しい説明変数として,パス遷移確率$P$を計測する.まず,対象ユーザ群から\ref{sec:before-analyze}節と同様に,各ノード間のCTパス重複数を算出する.
パス遷移確率$P_{n,m}$はノード$n$からノード$m$番目に到達する確率を意味し,式\ref{formula: single-ct-path}と式\ref{formula:ct-path}から算出する.ここで$N+1=\{ {n + 1}_1, {n + 1}_2,\ldots,{n + 1}_{k-1}, {n + 1}_k  \}$は,$n$の次に遷移するノード$n+1$の全体集合を意味する.

\vspace{-6mm}

\begin{equation}
  B_n = \frac{nからn+1のCT
  パス重複数}{\sum_{j=1}^{k} (nからn+1_jのCTパス重複数)} \label{formula: single-ct-path}
\end{equation}

\vspace{-5mm}

\begin{equation}\label{formula:ct-path}
  P_{n,m} = B_m \times B_{m-1} \times \ldots \times B_{n+1} \times P_n \quad 
\end{equation}

ユーザの作品制作過程をモデルに反映するため,ユーザが制作する$n$番目の作品から$m$番目までの各パス遷移確率の集合$P_{i,i+1}|(n \leq i \leq m - 1)=\{P_{n,n+1}, P_{n+1,n+2}, \dots, P_{m-2, m-1}, P_{m-1, m}\}$を全て計測する.また,作品$m$は対象ユーザ群のうち,各ユーザが最後に制作した作品のひとつ前の作品とし,$n-m$の作品間ノード数$L$がモデルに与える影響も調査するため,作品$n$には$\{m-1, m-2,\dots, m-(L-1), m-L\}$を順に代入し,$P_{i,i+1}|(n \leq i \leq m - 1)$を算出する.この時,対象ユーザの作品数がLに満たない場合,そのユーザはモデルの学習から除外することとする.

本実験には,\ref{sec:before-analyze}節で収集したユーザの作品を使用する.したがって,CT概念獲得有無と,パス遷移確率$P$を結合した説明変数を用いてBtoDモデル,DtoMモデルを構築する.

\subsection{実験結果}
表\ref{tab:btod-model-comp}は,従来BtoDモデルと提案BtoDモデルの中で最も精度の良かったモデル(L=1)の分類精度を示す.提案BtoDモデルは従来BtoDモデルと比べて適合率,再現率,F値において上回っているため,Developingに到達するユーザは直近に同じCT概念を持つ作品を制作することが多く,一度もDevelopingに到達しなかったユーザは直近に異なるCT概念を持つ作品を制作することが多いことが示唆される.

また,表\ref{tab:dtom-model-comp}は,従来DtoMモデルと提案Dtomモデルの中で最も精度の良かったモデル(L=1)の分類精度を示す.従来DtoMモデルと比べて提案DtoMモデルは適合率,F値において上回っているため,BtoDモデルと同様に,Masterに到達するユーザは直近に同じCT概念を持つ作品を制作することが多く,一度もMasterに到達しなかったユーザは直近に異なるCT概念を持つ作品を制作することが多いことが示唆される.

\begin{table}
  \caption{BtoDモデルの精度比較}
  \label{tab:btod-model-comp}
  \vspace{2mm}
  \centering
  \begin{tabular}{l|c|c|c}
    \hline
     & 適合率 & 再現率 & F1値\\
    \hline
    \hline
    従来BtoDモデル & 0.84 & 0.86 & 0.85\\
    \hline
    提案BtoDモデル(L=1) & \textbf{0.85} & \textbf{0.88} & \textbf{0.87}\\
    \hline
  \end{tabular}
\end{table}

\begin{table}
  \caption{DtoMモデルの精度比較}
  \label{tab:dtom-model-comp}
  \vspace{2mm}
  \centering
  \begin{tabular}{l|c|c|c}
    \hline
     & 適合率 & 再現率 & F1値 \\
    \hline
    \hline
    従来DtoMモデル & 0.70 & \textbf{0.44} & 0.54\\
    \hline
    提案DtoMモデル(L=1) & \textbf{0.79} & \textbf{0.44} & \textbf{0.57}\\
    \hline
  \end{tabular}
\end{table}

\subsection{考察}
BtoDユーザはある程度再現性のあるCTパスを通り,提案BtoDモデルを用いることで,作品制作過程を考慮した習熟度到達予測が可能となったが,作品制作数が多くなると作品制作過程が十分に考慮されないことが示された.
DtoMユーザは多様性のあるCTパスを通り,提案DtoMモデルを用いることで,作品制作数の多いユーザの作品制作過程を考慮した習熟度到達予測は可能となったが,作品制作数の少ないユーザは困難であることが示された.

\section{おわりに}

本研究では,Scratch上でのユーザのCTスキル獲得過程に基づく学習支援を目指し,ユーザが獲得してきたCT7概念の特徴量を分析した.分析の結果,BtoDユーザの多くは同一CT概念の作品を繰り返し制作し,DtoMユーザの多くは多様な作品を制作しつつも,制作作品数が多いユーザは共通したCT概念を持つ作品を制作することが多かった.これらの知見を基に,作品制作特性を反映した説明変数を用いたモデルを構築して従来モデルとの比較を行い,精度向上を実現した.本研究により,ScratchユーザのCTスキル獲得に合わせた学習支援の役立てとなることを期待する.

%%
%% 本文 - ここまで
%%

%%%%%%%%%%%%%%%%%%%%%%%%%%%%%%%%%%%%%%%%%%%%%%%%%%%%%%%%%%%%%%%%%%%%%%%%

%%
%% 参考文献
%%

\bibliographystyle{junsrt}
\bibliography{Okamoto}

%%%%%%%%%%%%%%%%%%%%%%%%%%%%%%%%%%%%%%%%%%%%%%%%%%%%%%%%%%%%%%%%%%%%%%%%

\end{document}
